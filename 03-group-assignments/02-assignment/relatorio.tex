\documentclass[11pt]{article}

\usepackage[a4paper, total={6.5in, 9.5in}]{geometry}
%\usepackage[T1]{fontenc}
%\usepackage[utf8]{inputenc}
\usepackage[portuguese]{babel}
\usepackage{graphicx}
%\usepackage{ae,aecompl,aeguill}
%\usepackage{showframe}% just to show page geometry, not needed
\setlength{\parskip}{1mm plus 4mm minus 3mm}

\begin{document}

% ========== Edit your name here
\author{
    Rodrigo Orem\\
    \texttt{8921590}
    \and
    Édio Cerati\\
    \texttt{9762678}
    \and
    Thiago Teixeira\\
    \texttt{10736987}
}

\title{\vspace{-2em}Relatório do EP2 -- Mandelbrot}
\maketitle

Neste EP, desenvolvemos um algoritmo para gerar imagens fractais baseado
no conjunto de Mandelbrot. Então, parelizamos em OpenMP e CUDA para
comparar os ganhos de desempenho.

Nossa primeira versão do algoritmo, sequencial, não oferecia ganhos
significativos de desempenho ao paralelizar com OpenMP. Percebemos que
isso ocorria porque estávamos usando um número baixo para M, que é o
número máximo de iterações usado para definir se um ponto pertence ou
não a um conjunto. No caso, se $|z_j∣ \leq 2$ para todo $j < M$, então o
ponto $z_j$ pertence ao conjunto e deve ser pintado de preto.

Começamos usando $M=50$, mas paralelizar esse número não oferecia ganhos
de desempenho: em alguns casos até ficava mais lento, devido ao
overhead. Quando aumentamos para $M=1000$, obtivemos resultados
significativos de ganho de desempenho.

A tabela a seguir mostra os resultados de 20 execuções do programa no
servidor Dota da Rede Linux. Os resultados estão disponíveis na pasta
{\tt test/}.

\begin{table}[h!]
\centering
\begin{tabular}{lll}
\hline
\multicolumn{1}{c}{Threads} &
\multicolumn{1}{l}{Média Tempo} &
\multicolumn{1}{l}{Desvio Padrão} \\
\hline
1 & 10,10 & 0,128 \\
8 & 1,51 & 0,040 \\
16 & 0,94 & 0,030
\end{tabular}
\end{table}

Como podemos ver pela tabela, ao utilizar OpenMP conseguimos executar o
programa 10,7 vezes rápido ao usar 16 threads.

\end{document}
